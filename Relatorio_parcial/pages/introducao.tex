\section{Introdução e Motivação}

A dinâmica dos fluidos é uma área na qual tendem a surgir sistemas de equações bastante complexas e, por isso, desde o início do uso de computadores para simulações, a Dinâmica dos Fluidos Computacional (CFD) se fez uma das suas grandes vertentes de estudo.
Atualmente, com uma grande expansão do uso de Inteligência Artificial em diversos setores, naturalmente são aplicados diversos métodos de aprendizado de máquina na área de CFD. Nesse contexto, um dos métodos que se destacam são as Redes Neurais Artificiais, em especial as de \textit{Multi-Layer Perceptron} \cite{Sharma2023-fr}.

Nessa pesquisa, são utilizadas redes neurais do tipo \textit{multi-layer perceptron} para aperfeiçoar métodos de soluções numéricas aplicados em equações de dinâmica dos fluidos e futuramente analisar, até certo escopo, o tempo de execução dos algoritmos, a precisão das soluções encontradas e outros parâmetros relevantes para a escolha do uso do método.

\subsection{Mudança de direcionamento da pesquisa}

No início da pesquisa bibliográfica, o foco era o estudo da implementação de redes neurais para o aprimoramento de solução de sistemas por meio do Método de Diferenças Finitas.
Porém, após um estudo inicial desse método (baseado em \citeonline{Langtangen2017-pd}) e do Método de Elementos Finitos (a partir de \citeonline{Becker1981-dz}), da formação de um grupo de estudo com professores e alunos de pós-graduação focados na biblioteca \texttt{FEniCSx} (que se baseia em elementos finitos), uma decisão foi feita para uma mudança na metodologia do projeto.
Por contar com o apoio dos integrantes do grupo de estudos de elementos finitos, e após a constatação de que a mudança de foco no estágio inicial do projeto seria possível, a alteração foi considerada uma decisão sensata pelo aluno e pelo orientador da pesquisa.

Tanto o de diferenças finitas, quanto o de elementos finitos, são métodos de solução numérica extensivamente aplicados e consolidados historicamente no contexto de equações de dinâmica dos fluidos \cite{Thomee1984-sc} e, no contexto de aprendizado de máquina, as redes neurais artificiais já se provaram como uma ferramenta capaz de aperfeiçoar ambos os métodos, como se pode verificar em \citeonline{Pantidis2023-vh, Meethal2023-ag, Le-Duc2023-ly} para elementos finitos e \citeonline{Tu2022-hk, Shi2020-wm} para diferenças finitas.
Assim, a mudança da metodologia da pesquisa ainda conserva o uso de métodos já consolidados para soluções numéricas.