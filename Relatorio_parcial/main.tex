\documentclass[
	% -- opções da classe memoir --
	12pt,				% tamanho da fonte
	openany,			% capítulos começam em pág ímpar (insere página vazia caso preciso)
	oneside,				% para impressão em verso e anverso. Oposto a oneside
	openright, 
    a4paper,				% tamanho do papel. 
	article,
	% -- opções da classe abntex2 --
	%chapter=Title,		% títulos de capítulos convertidos em letras maiúsculas
	%section=TITLE,			% títulos de seções convertidos em letras maiúsculas
	%subsection=TITLE,		% títulos de subseções convertidos em letras maiúsculas
	%subsubsection=TITLE,	% títulos de subsubseções convertidos em letras maiúsculas
	% -- opções do pacote babel --
	english,			% idioma adicional para hifenização
	french,			% idioma adicional para hifenização
	spanish,			% idioma adicional para hifenização
	brazil			% o último idioma é o principal do documento
	]{abntex2}

% ---
% Pacotes básicos 
% ---
\usepackage{lmodern}			% Usa a fonte Latin Modern			
\usepackage[T1]{fontenc}			% Selecao de codigos de fonte.
\usepackage[utf8]{inputenc}		% Codificacao do documento (conversão automática dos acentos)
\usepackage{lastpage}			% Usado pela Ficha catalográfica
\usepackage{indentfirst}			% Indenta o primeiro parágrafo de cada seção.
\usepackage[table,dvipsnames]{xcolor}				% Controle das cores
\usepackage{graphicx}				% para lidar com gráficos
\graphicspath{{./Figuras/}}
\DeclareGraphicsExtensions{.pdf,.jpeg,.jpg,.png}
\usepackage{microtype} 			% para melhorias de justificação
\usepackage{amsmath}
\usepackage{mathtools}
\usepackage{amsfonts}
\usepackage{bm}
\usepackage{enumitem}
\usepackage{siunitx}
\usepackage{multirow}
\usepackage{caption}
\usepackage[labelformat=simple]{subcaption}
\usepackage{xcolor}
\usepackage{booktabs}
\usepackage{pdfpages}
\usepackage{alphalph}
\usepackage{xfrac}
\usepackage{epigraph}
% ---

% ---
% Pacotes adicionais, usados apenas no âmbito do Modelo Canônico do abnteX2
% ---
\usepackage{lipsum}				% para geração de dummy text
% ---

% ---
% Pacotes de citações
% ---
\usepackage[brazilian,hyperpageref]{backref}	% Paginas com as citações na bibl
%\usepackage[alf,abnt-etal-list=0,abnt-etal-cite=3,bibjustif,abnt-thesis-year=both]{abntex2cite}					% Citações padrão ABNT
\usepackage[alf,abnt-repeated-title-omit=yes,abnt-emphasize=bf,abnt-etal-list=0]{abntex2cite}
\usepackage{IEEEtrantools}
%\begin{IEEEeqnarray}{rCl}
%	z(t) & = & 0  \\
%	t_i(t) & = & 0
%\end{IEEEeqnarray}

% *** ALIGNMENT PACKAGES ***
\usepackage{array}
\newcolumntype{L}[1]{>{\raggedright\let\newline\\\arraybackslash\hspace{0pt}}p{#1}}
\newcolumntype{C}[1]{>{\centering\let\newline\\\arraybackslash\hspace{0pt}}p{#1}}
\newcolumntype{R}[1]{>{\raggedleft\let\newline\\\arraybackslash\hspace{0pt}}p{#1}}

% --- 
% CONFIGURAÇÕES DE PACOTES
% --- 

% ---
% Configurações do pacote backref
% Usado sem a opção hyperpageref de backref
\renewcommand{\backrefpagesname}{}
% Texto padrão antes do número das páginas
\renewcommand{\backref}{}
% Define os textos da citação
\renewcommand*{\backrefalt}[4]{}%
% ---

% ---
% Configurações do pacote siunitx
\sisetup{
   output-decimal-marker = {,} ,
   group-separator = {.}
}
% ---
% Configurações do pacote subcaption
%\renewcommand\thesubfigure{(\alph{subfigure})}
\renewcommand*{\thesubfigure}{(\alphalph{\value{subfigure}})}
% ---

% Configurações do pacote epigraph
\setlength{\epigraphrule}{0pt}

% ---
% Definições de comandos
% ---
\newcommand{\myVec}[1]{\bm{#1}}
\newcommand{\myHat}[1]{\hat{\bm{#1}}}
\newcommand{\myMatrix}[1]{\bm{\mathit{#1}}}
\newcommand{\ud}{\,\mathrm{d}}
\newcommand{\tabLabel}[1]{\multicolumn{1}{c}{\textbf{#1}}}
\newcommand{\tabLabelMrow}[2]{\multicolumn{1}{c}{\multirow{#1}{*}{\textbf{#2}}}}
\newcommand{\rowSpace}[1]{\renewcommand{\arraystretch}{#1}}
\newcommand{\minitab}[2][c]{\begin{tabular}{#1}#2\end{tabular}}

\newlist{sumariotese}{enumerate}{4}
\setlist[sumariotese,1]{label=\arabic*}
\setlist[sumariotese,2]{label=\arabic{sumariotesei}.\arabic*}
\setlist[sumariotese,3]{label=\arabic{sumariotesei}.\arabic{sumarioteseii}.\arabic*}
\setlist[sumariotese,4]{label=\arabic{sumariotesei}.\arabic{sumarioteseii}.\arabic{sumarioteseiii}.\arabic*}

% ---
% Informações de dados para CAPA e FOLHA DE ROSTO
% ---
\titulo{Estudo e implementação de redes neurais hierárquicas de aprendizado profundo para cálculo de interpolações em elementos finitos}
\autor{Pedro Azevedo Coelho Carriello Corrêa}
\local{São Carlos, SP}
\data{Março, 2024}
\orientador{Dr. Pablo Giovanni Silva Carvalho}
%\coorientador{Equipe \abnTeX}
\instituicao{Escola de Engenharia de São Carlos}
\tipotrabalho{Relatório de PIBIC}
% O preambulo deve conter o tipo do trabalho, o objetivo, 
% o nome da instituição e a área de concentração 
%\preambulo{Relatório de Projeto de Iniciação Científica para cumprir com os requisitos estabelecidos no Edital 081/2019-PROPESP/UFAM do Programa Institucional de Bolsas de Iniciação Científica da Universidade Federal do Amazonas.}
% ---


% ---
% Configurações de aparência do PDF final

% alterando o aspecto da cor azul
\definecolor{blue}{RGB}{41,5,195}

% informações do PDF
\makeatletter
\hypersetup{
     	%pagebackref=true,
	pdftitle={\@title}, 
	pdfauthor={\@author},
    	pdfsubject={\imprimirpreambulo},
	pdfcreator={LaTeX with abnTeX2},
	%pdfkeywords={ufam}{ft}{engenharia mecânica}{mecânica do contínuo}, 
	colorlinks=true,       		% false: boxed links; true: colored links
    	linkcolor=black,          	% color of internal links
    	citecolor=black,        		% color of links to bibliography
    	filecolor=black,      		% color of file links
	urlcolor=black,
	bookmarksdepth=4
}
\makeatother
% --- 

% --- 
% Espaçamentos entre linhas e parágrafos 
% --- 

% O tamanho do parágrafo é dado por:
\setlength{\parindent}{1.3cm}

% Controle do espaçamento entre um parágrafo e outro:
\setlength{\parskip}{0.2cm}  % tente também \onelineskip

% Espessura das bordas das tabelas
\setlength\heavyrulewidth{0.12em}

% ---
% compila o indice
% ---
\makeindex
% ---

% NEW COMMANDS
\providecommand{\en}[1]{\ensuremath{\textrm{\textsc{e}-}{#1}}}
\providecommand{\ep}[1]{\ensuremath{\textrm{\textsc{e}}{#1}}}

\renewcommand{\imprimircapa}{
	\begin{capa}
		\begin{center}
		\begin{tabular}{L{0.25\textwidth}C{0.5\textwidth}R{0.25\textwidth}}
			\multirow{3}{*}{\includegraphics[width=0.25\textwidth]{./figures/Logo-CNPq.png}} &
			{\large\ABNTEXchapterfont UNIVERSIDADE DE SÃO PAULO} \vspace*{0.2cm} & 
            \multirow{3}{*}{\includegraphics[width=0.25\textwidth]{./figures/Logo-EESC.png}} \\
			& {\large\ABNTEXchapterfont ESCOLA DE ENGENHARIA DE SÃO CARLOS} \vspace*{0.2cm} & \\
		\end{tabular}
		\\[0.5cm]
		%\includegraphics[width=3cm]{./figures/Logo-EESC.png} \\
		{\large Programa Institucional de Bolsas de Iniciação Científica (PIBIC)} \\[4cm]
		{\bfseries\ABNTEXchapterfont\LARGE\MakeUppercase Relatório Parcial} \\[1cm]
		{\bfseries\ABNTEXchapterfont\LARGE\imprimirtitulo} \\[4.5cm]
		{\large\ABNTEXchapterfont Autor: \imprimirautor} \\
        {\large\ABNTEXchapterfont Orientador: \imprimirorientador}
		\vfill
		{\large\ABNTEXchapterfont\imprimirlocal} \\[0.2cm]
		{\large\ABNTEXchapterfont\imprimirdata}
		\vspace*{1cm}
		\end{center}
	\end{capa}
}

% ----
% Início do documento
% ----
\begin{document}

% Retira espaço extra obsoleto entre as frases.
\frenchspacing 

% ----------------------------------------------------------
% ELEMENTOS PRÉ-TEXTUAIS
% ----------------------------------------------------------
% \pretextual

% ---
% Capa
% ---
\imprimircapa
% ---

% ---
% Folha de rosto
% (o * indica que haverá a ficha bibliográfica)
% ---
%\imprimirfolhaderosto*

\textual

\section{Introdução e Motivação}

A dinâmica dos fluidos é uma área na qual tendem a surgir sistemas de equações bastante complexas e, por isso, desde o início do uso de computadores para simulações, a Dinâmica dos Fluidos Computacional (CFD) se fez uma das suas grandes vertentes de estudo.
Atualmente, com uma grande expansão do uso de Inteligência Artificial em diversos setores, naturalmente são aplicados diversos métodos de aprendizado de máquina na área de CFD. Nesse contexto, um dos métodos que se destacam são as Redes Neurais Artificiais, em especial as de \textit{Multi-Layer Perceptron} \cite{Sharma2023-fr}.

\subsection{Mudança de direcionamento da pesquisa}
No início da pesquisa bibliográfica, o foco era o estudo da implementação de redes neurais para o aprimoramento de solução de sistemas por meio do Método de Diferenças Finitas.
Porém, após um estudo inicial desse método (baseado em \citeonline{Langtangen2017-pd}) e do Método de Elementos Finitos (a partir de \citeonline{Becker1981-dz}), da formação de um grupo de estudo com professores e alunos de pós-graduação focados na biblioteca \texttt{FEniCSx} (que se baseia em elementos finitos), uma decisão foi feita para uma mudança na metodologia do projeto.
Por contar com o apoio dos integrantes do grupo de estudos de elementos finitos, e após a realização que a mudança de foco no estágio inicial do projeto seria possível, a alteração foi considerada uma decisão sensata pelo aluno e pelo orientador da pesquisa.

Tanto o de diferenças finitas, quanto o de elementos finitos, são métodos de solução numérica extensivamente aplicados e consolidados historicamente no contexto de equações de dinâmica dos fluidos \cite{Thomee1984-sc} e, no contexto de aprendizado de máquina, as redes neurais artificiais já se provaram como uma ferramenta capaz de aperfeiçoar ambos os métodos, como se pode verificar em \citeonline{Pantidis2023-vh, Meethal2023-ag, Le-Duc2023-ly} para elementos finitos e \citeonline{Tu2022-hk, Shi2020-wm} para diferenças finitas.
Assim, a mudança da metodologia da pesquisa ainda conserva o uso de métodos já consolidados para soluções numéricas.

\subsection{Redes Neurais \textit{Multi-Layer Perceptron}}

\subsection{Método de Elementos Finitos}
\section{Objetivos}

Inspirado nos artigos \citeonline{Saha2021-kw} e \citeonline{Zhang2021-ln}, o projeto visa a formulação de um método que utiliza uma rede neural hierárquica para realimentar a localização dos nós de elementos finitos. 
Essa realimentação viria a partir do treinamento da rede e, a partir dela, os elementos finitos ficariam mais refinados nas regiões de maior sensibilidade da solução, ou seja, os nós iriam se concentrar nas regiões que a solução mais flutua, de modo a diminuir seu erro.
\section{Metodologia}

Os principais modelos de redes neurais e métodos de aprendizado de máquina foram estudados pelo aluno por meio do curso das disciplinas SCC0230 - Inteligência Artificial e SCC0270 - Redes Neurais e Aprendizado Profundo, ministradas, respectivamente, pela Prof. Solange Oliveira Rezende e Prof. Moacir Antonelli Ponti.
Já o método de elementos finitos foi estudado principalmente a partir da leitura de \citeonline{Becker1981-dz}. 

Já a implementação do algoritmo está sendo em \texttt{Python}, principalmente através do uso da biblioteca \texttt{PyTorch}. Também estão sendo utilizadas outras bibliotecas auxiliares, como a \texttt{MatPlotLib} e \texttt{NumPy}.

\subsection{Redes Neurais \textit{Multi-Layer Perceptron}}

Redes neurais de aprendizado profundo são redes que conseguem, em grande parte, superar limitações de modelos lineares \textit{perceptron} incorporando mais camadas. 
A forma mais natural de fazer isso é alocando camadas conectadas em seguida, de forma que cada camada anterior alimenta a próxima, até gerar a saída do modelo. 
Essa arquitetura é chamada \textit{multilayer perceptron} (MLP) \cite{Zhang2021-od}.

No geral, redes desse tipo são modelos de aprendizado supervisionado, ou seja, que são treinados a partir de um banco de dados de entradas e suas respectivas saídas esperadas. 
Porém, nesse projeto foi feita uma modificação em relação ao modelo \textit{perceptron} original, na qual a função de perda (\textit{Loss Function}) utilizada para treinar o modelo não é calculada a partir de um banco de dados, mas sim, recalculando a nova solução em elementos finitos considerando as novas posições dos nós dadas pela rede e retornando o seu erro na equação original. 
Tal abordagem é mais explicada na Seção \ref{sec:calculo_da_perda}.

\subsection{Método de Elementos Finitos}

Atualmente, a equação genérica governante no domínio $ x_0 < x < x_L $ que o algoritmo em desenvolvimento resolveria numericamente tem a seguinte forma:

\begin{equation}
    -k \frac{\mathrm{d} u(x)^2 }{ \mathrm{d}^2 x} + c \frac{\mathrm{d} u(x)}{\mathrm{d} x} + b u(x) = f(x)
    \label{eq:equacao_generica}
\end{equation}

Onde $k$, $c$ e $b$ são coeficientes fixos e $f(x)$ uma função de $x$ dados pela equação que se queira resolver. 
Considerando condição de borda de Dirichlet onde $u(x_0)=u_0$ e $u(x_L)=u_L$, e $v(x)$ uma função de teste definida em todo o intervalo, pode-se analisar a Equação \ref{eq:equacao_generica} pelo ponto de vista do método de elementos finitos e reescrevê-la na forma variacional:

\begin{equation}
    \int_{x_0}^{x_L} ( ku' v' + cu' v + buv ) \mathrm{d}x = \int_{x_0}^{x_L} (fv) \mathrm{d}x
    \label{eq:equacao_generica_variacional}
\end{equation}

Nesse contexto, $v$ e $u$ para satisfazerem corretamente o enunciado devem pertencer ao subespaço $H^1$, que são as funções cujas integrais da Equação \ref{eq:def_H1} convergem. 

\begin{equation}
    \int_{x_0}^{x_L} \left[ (v')^2 + v^2 \right] \mathrm{d} x < + \infty
    \label{eq:def_H1}
\end{equation}

Dividi-se, arbitrariamente, o domínio do problema em $N$ elementos finitos com $h$ de comprimento. 
Após isso, são construídas uma função de forma $\phi_i$ para cada elemento e que geram uma base para o subespaço $H^h$ de $H^1$: 
$\{ \phi_1, \phi_2, \dots , \phi_N \}$.

Assim, procura-se uma função $u_h \in H_h$ que pode ser escrita como:

\begin{equation}
    u_h(x) = \sum_{j=1}^{N} \alpha_j \phi_j(x)
    \label{eq:def_u_h}
\end{equation}

Escrevendo $v_h$ da mesma forma e substituindo na Equação \ref{eq:equacao_generica_variacional}, temos:

\begin{equation*}
    \int_{x_0}^{x_L} ( k u_h' v_h' + c u_h' v_h + b u_h v_h ) \mathrm{d} x = \int_{x_0}^{x_l} (fv_h) \mathrm{d} x
\end{equation*}

Equivalentemente:

\begin{equation}
    \sum_{j=1}^{N} K_{ij} \alpha_j = F_i, \quad i=1,2, \dots N
    \label{eq:K*alpha=F}
\end{equation}

Sendo $K$ comumente chamada de matriz de rigidez e $F$, de vetor de carga.  

\begin{align}
    K_{ij} &= \int_{x_0}^{x_L} ( k \phi_i' \phi_j' + c \phi_i' \phi_j + b \phi_i \phi_j ) \mathrm{d} x \\
    F_i &= \int_{x_0}^{x_L} ( f \phi_i ) \mathrm{d} x
\end{align}

Com $1 \leq i,j \leq N$. 

Após resolver a Equação \ref*{eq:K*alpha=F} para os coeficientes $\alpha_i$, utilizamos a Equação \ref*{eq:def_u_h} para obter a aproximação de Galerkin para o problema. Todo esse processo está descrito em \citeonline{Becker1981-chapter2}.

Atualmente, o código do projeto utiliza esse procedimento para encontrar a primeira iteração da aproximação da solução e, após o treinamento da rede, a ideia era que os nós se deslocassem para posições mais convenientes e as matrizes $K$ e $F$ fossem recalculadas considerando as novas funções de forma dos nós.
Porém, estão ocorrendo problemas com a implementação do código com a biblioteca \texttt{PyTorch} e os nós ainda não se delocam.

Algo interessante é que há infinitas possibilidades de escolhas para a função utilizada como $\phi_i$, o que torna esse método bastante flexível. 
Até então, o modelo apenas trabalha com uma função de forma linear, mas há trabalho para futuramente adicionar também a opção de aproximações quadráticas.

A função de forma linear é definida de acordo com a Equação \ref{eq:def_linear_funcaoforma}, considerando $h_i$ o tamanho de cada elemento finito, que agora pode ser variável.

\begin{equation}
    \phi_i(x)= 
    \begin{dcases}
        \frac{x-x_i}{h_i}, \quad & \text{para } x_{i-1} \leq x \leq x_i \\
        \frac{x_{i+1}-x}{h_{i+1}}, \quad & \text{para } x_{i} \leq x \leq x_{i+1} \\
        0, \quad & \text{para } x \leq x_{i-1} \text{e } x \geq x_{i+1}
    \end{dcases}
\label{eq:def_linear_funcaoforma}
\end{equation}

\subsection{Redes Neurais de Aprendizado Profundo Hierárquicas (HiDeNN)} 

\subsection{Cálculo da perda para treinamento do modelo} \label{sec:calculo_da_perda}
\input{pages/atividades}
\input{pages/resultados}
\section{Próximas Etapas}

\begin{itemize}
    \item consertar treinamento
    \item utilizar a rede tbm para recalcular os displacements
    \item readequar o codigo para rodar a partir da biblioteca FEniCSx
    \item utilizar funcoes de forma quadraticas tbm
\end{itemize}


\postextual

\bibliography{referencias.bib}

\end{document}